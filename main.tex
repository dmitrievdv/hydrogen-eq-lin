\documentclass{article}
\usepackage[utf8]{inputenc}
\usepackage[english, russian]{babel}
\usepackage[section]{placeins}
%\usepackage{upgreek}
\usepackage{graphicx}
\usepackage{caption}
%\usepackage{subcaption}
\usepackage{wrapfig}
\usepackage{amsmath}
\usepackage{amssymb}
\usepackage{gensymb}
%\usepackage{cyr}
\usepackage{epsf}

\usepackage[top=20mm, bottom=20mm, left=20mm, right=20mm]{geometry}

\title{How many levels is needed in non-stationary regime?}

\begin{document}
\maketitle
    
\section{Sources of hydrogen atom levels populations in 1D case}

Sources of populations for each hydrogen energy level are written as
\begin{equation}\label{eq:sources}
    \begin{aligned}
        \sigma_i &= \sum\limits_{k=i+1}^\infty n_k(A_{ki} + B_{ki}J_{ki}) + \sum\limits_{j=1}^{i-1} n_jB_{ji}J_{ij}+ n_e^2(A_{ci} + WB_{ci})+\\
        &+n_e\left(\sum\limits_{j\neq i}n_jq_{ji}  + n_e^2q_{ci}\right)-\\
        &-n_i\left(\sum\limits_{j=1}^{i-1}(A_{ij} + B_{ij}J_{ij}) + \sum\limits_{k=i+1}^\infty B_{ik}J_{ki} + WB_{ic}\right) -\\
        &-n_in_e\left(q_{ic} + \sum\limits_{i \neq j}q_{ij}\right)
    \end{aligned}
\end{equation}
It's convinient to introduce so called Menzel parameters \(b_i\), defined by
\begin{equation}\label{eq:menzel}
    \frac{n_i}{n_e^2} = b_i\frac{i^2h^3}{(2\pi m_ek_BT_e)^{3/2}}e^{X_i},
\end{equation}
where \(X_j = 13.6\ \mathrm{eV}/(i^2k_BT_e)\). Substituting \(b_i\) in \eqref{eq:sources} and using the relations between collisonal coefficients
\[
\frac{q_{ci}}{q_{ic}} = \frac{i^2h^3}{(2\pi m_ek_BT_e)^{3/2}}e^{X_i}\ \mathrm{and}\  \frac{q_{ul}}{q_{lu}} = \frac{l^2}{u^2}e^{X_l - X_u}
\]
\begin{equation}
    \begin{aligned}
        \frac{(2\pi m_ek_BT_e)^{3/2}}{n_e^2i^2h^3}e^{-X_i}\sigma_i &= \sum\limits_{k=i+1}^\infty b_k\frac{k^2}{i^2}e^{X_k-X_i}(A_{ki} + B_{ki}J_{ki}) + \sum\limits_{j=1}^{i-1} b_j\frac{j^2}{i^2}e^{X_j-X_i}B_{ji}J_{ij} +\\
        &+\frac{(2\pi m_ek_BT_e)^{3/2}}{i^2n_e^2h^3}e^{-X_i}(A_{ci} + WB_{ci}) +n_e\left(\sum\limits_{j \neq i}(b_j-b-i)q_{ij}  + q_{ci}\right)-\\
        &-b_i\left(\sum\limits_{j=1}^{i-1}(A_{ij} + B_{ij}J_{ij}) + \sum\limits_{k=i+1}^\infty B_{ik}J_{ki} + WB_{ic}\right) -\\
        &-b_in_e\left(q_{ic} + \sum\limits_{i \neq j}q_{ij}\right)
    \end{aligned}
\end{equation}
Then, applying Sobolev approximation,
\[
J_{ul} = \frac{2h\nu_{ul}^3}{c^2}\left(\overline{S}_{ul}(1-\beta_{lu}) + W\beta_{lu} \overline{I^\star}_{ul}\right),
\]
assuming CFR
\[
\overline{S}_{ul} = \left(\frac{n_l}{n_u}\frac{u^2}{l^2} - 1\right)^{-1} = \left(\frac{b_l}{b_u}e^{X_l-X_u} - 1\right)^{-1}
\]
and black-body star
\[
\overline{I^\star}_{ul} = \left(e^{X_l - X_u} - 1\right)^{-1}
\]
and using relations between einstein coefficients
\[
A_{ul} = \frac{2h\nu_{ul}^3}{c^2}B_{ul}\ \mathrm{and}\ \frac{B_{ul}}{B_{lu}} = \frac{l^2}{u^2},
\]
we get
\begin{equation}
    \begin{aligned}
        \frac{(2\pi m_ek_BT_e)^{3/2}}{n_e^2i^2h^3}e^{-X_i}\sigma_i &= \sum\limits_{k=i+1}^\infty b_k\frac{k^2}{i^2}e^{X_k-X_i}A_{ki}(\beta_{ik} + W\overline{I^\star}_{ki}\beta_{ik}) + \sum\limits_{j=1}^{i-1} b_je^{X_j-X_i}A_{ij}W\beta_{ji} \overline{I^\star}_{ij} +\\
        &+\frac{(2\pi m_ek_BT_e)^{3/2}}{i^2n_e^2h^3}e^{-X_i}(A_{ci} + WB_{ci}) +n_e\left(\sum\limits_{j \neq i}(b_j-b_i)q_{ij}  + q_{ci}\right)-\\
        &-b_i\left(\sum\limits_{j=1}^{i-1}A_{ij}(\beta_{ji} + W\overline{I^\star}_{ij}\beta_{ji}) + \sum\limits_{k=i+1}^\infty A_{ki}\frac{k^2}{i^2}W\overline{I^\star}_{ki}\beta_{ik}  + WB_{ic}\right).
    \end{aligned}
\end{equation}

\section{Sources linearization}

\end{document}
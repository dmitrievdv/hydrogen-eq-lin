\documentclass{article}
\usepackage[utf8]{inputenc}
\usepackage[english, russian]{babel}
\usepackage[section]{placeins}
%\usepackage{upgreek}
\usepackage{graphicx}
\usepackage{caption}
%\usepackage{subcaption}
\usepackage{wrapfig}
\usepackage{amsmath}
\usepackage{amssymb}
\usepackage{gensymb}
%\usepackage{cyr}
\usepackage{epsf}

\title{How many levels is needed in non-stationary regime?}

\begin{document}
\maketitle
    
\section{Stationary equations for 1D case}

Sources of populations for each hydrogen energy level are written as
\begin{equation}\label{eq:sources}
    \begin{aligned}
    \sigma_i = & \left[\sum\limits_{k=i+1}^\infty n_k (A_{ki} + B_{ki}J_{ki}) + \sum\limits_{j=1}^{i-1} n_jB_{ji}J_{ij} +  n_e \sum\limits_{j\neq i}^\infty n_jq_{ji}\right] -  \\ 
    -n_i & \left[ \sum\limits_{j=1}^{i-1} (A_{ij} + B_{ij}J_{ij}) + \sum\limits_{k=i+1}^\infty B_{ik}J_{ik} + n_e ( q_{ic} + \sum\limits_{j \neq i}^\infty q_{ij} ) + B_{ic}WJ_{ic}^\star \right] +  \\
     +& n_e^2C_i + n_e^3Q_{ci},\ \ i=1,\ 2,\ 3... 
    \end{aligned}
\end{equation}
The line intensity \(J_{ki}\) can be calculated using Sobolev's approximation
\[
J_{ik} = S_{ik}(1 - \beta_{ki}) + WI_{ik}\beta_{ki},
\]
where
\[
S_{ik} = \frac{2h\nu_{ik}^3}{c^2}\left(\frac{n_kg_i}{n_ig_k} - 1\right)^{-1},
\]
\[
I_{ik} = \frac{2h\nu_{ik}^3}{c^2}\left(e^{\frac{h\nu_{ik}}{k_BT_s}} - 1\right)^{-1}.
\]
\(\beta_{ki}\) is the photon escape propability
\[
\beta_{ki} = \frac{1-e^{\tau_{ki}}}{\tau_{ki}},
\]
\[
\tau_{ki} = \frac{\pi e^2 f_{ik} z}{m_e \nu_{ik} v} n_i \left(1 - \frac{n_kg_i}{n_ig_k}\right).
\]
Equations \eqref{eq:sources} can be simplified using Menzel parameters \(b_i\), defined by
\begin{equation}\label{eq:menzel}
    \frac{n_i}{n_e^2} = b_i\frac{i^2h^3}{(2\pi m_ek_BT_e)^{3/2}}e^{X_i},
\end{equation}
where \(X_j = 13.6\ \mathrm{eV}/(i^2k_BT_e)\). Substituting \eqref{eq:menzel} in \eqref{eq:sources}


\end{document}